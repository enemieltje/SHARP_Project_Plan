\section{Background}

% \textit{
%     This chapter should contain a description of the company,
%     who they are and why do they want to do the project.
%     Other stakeholders should also be mentioned and what interest they have in the project.
%     A brief overview of the report structure can also be given here.
% }

% \begin{itemize}
%     \item Talk about SSI, the development of (atmosphere breathing) propulsion systems for small satellites.
%     \item Development state of the valve and why testing it in a relevant environment is important.
%     \item Solid Flow and their CGG's which together with the valve can be a weight efficient fuel delivery system.
%     \item This would make for a nice aviation energy transition project, hence the drone.
%     \item Internship is only a part of SHARP, focusing on developing the H2 power system.
%     \item This report is made in accordance with the guidelines in the book '\emph{Project Management}'~\cite{Grit2011}.
% \end{itemize}

% \textit{
%     This leads to testing the valve being the main goal.
%     An alternative is talking about how batteries are too heavy for electric flight and have aviation energy transition as main goal.
%     \newline
%     % \\Do I also include reference projects such as the NLR drone, or is that in literature report later?
%     \\Problem statement:
%     \\Batteries are too heavy for electric flight, OR,
%     \\Need an affordable way to test the valve in a relevant environment
% }

\subsection*{Stellar Space Industries}
Stellar Space Industries (SSI) is a Dutch aerospace company located in Noordwijk near ESA's ESTEC facility.
SSI specialises in high precision manufacturing capabilities
and covers the entire product development cycle including design, machining, assembly, integration and testing.
Their production facility is equipped with high-precision machining and testing equipment.

In collaboration with ESA, SSI is developing a novel electrodeless electric propulsion system for small satellites.
This propulsion system aims to increase efficiency, lifespan and reliability compared to current state-of-the-art electric propulsion systems.
SSI is currently in their second phase of development where they aim to design build and test the entire system in-house.

Presently, several subsystems of the propulsion system are being developed and tested concurrently.
During the development of these subsystems, SSI is also looking for opportunities to apply the technology of each component in other applications.
One of the subsystems that is currently ready for testing and application is SSI's flow control unit (FCU) which is a high-precision valve that can control the mass flow of gaseous propellants.


\subsection*{Project SHARP}
In order to find a relevant application for their FCU, SSI has initiated the Solid Hydrogen Aircraft Regulated Propulsion (SHARP) project.
SHARP is a joint investigation between SSI and Solid Flow, a Dutch company specialising in the development of solid hydrogen storage systems.
The goal of SHARP is to investigate the feasibility of hydrogen-fuelled drones that utilise solid-state hydrogen storage.
In this collaboration, Solid Flow B.V. provides the cool gas generator (CGG), a system capable of producing gaseous hydrogen on demand from solid hydrogen storage.
Stellar Space Industries B.V. contributes its expertise in flow control systems for electric propulsion, which may be adapted for hydrogen-based applications.

SHARP is conducted under the Luchtvaart in Transitie (LIT) program of the RVO of the Dutch government, which aims to advance sustainable aviation technologies. SHARP aims to contribute to the aviation energy transition by providing a scalable, weight and volume efficient energy source that is CO\textsubscript{2} and NO\textsubscript{x} neutral. SHARP first aims to demonstrate the feasibility of the technology in drone applications, allowing future projects to scale the technology to larger aircraft and replace traditional jet fuel-based propulsion systems.

\subsection*{SHARP --- Power Generation and Delivery System}
This document considers the project plan for a subsystem of the SHARP project, specifically the development of the power generation and delivery system. This part of the SHARP project is conducted as a graduation internship. Throughout this document the part of SHARP that falls under the graduation internship will be referenced as SHARP --- Power System (SHARP-PS).

\newpage
