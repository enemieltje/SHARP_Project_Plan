\section{Project Results}
For Stellar Space Industries, the primary goal for SHARP is to test and apply their FCU in a real world scenario and thus advance the FCU's technology readiness level (TRL) from 4 to 6. In addition to this, they aim to find a secondary market for their FCU in hydrogen based aviation compared to electric propulsion for satellites. Solid Flow has the same goals for their hydrogen CGG.
The goal of SHARP-PS is to integrate the FCU and CGG and demonstrate the technologies and their compatibility in a relevant environment.

The relevant evnironment for the SHARP-PS project will be a hydrogen fuel cell that delivers power to an unmanned areal vehicle (UAV). The results of the SHARP-PS project will be the following:

\begin{itemize}
    \item The design for a power generation and feed system that is integrated into a UAV.
    \item A test bench setup that validates the performance of this power generation and feed system.
    \item A research report that documents the design process and test results and assesses the feasibility of using this power system in UAV's.
\end{itemize}

If allowed by time, a UAV with the integrated power generation and feed system may be an additional result of the SHARP-PS project. This will be described in more detail in section 4 Scope.

% The relevant environment for the SHARP-PS project will be a fuel cell that delivers power to a motor and propeller system. Results of tests performed in this environment will be directly applicable to hydrogen based aviation projects. The result of the SHARP-PS project will be a test bench setup that includes this fuel cell and motor-propeller system and the integrated FCU and CGG hydrogen supply. This test bench setup shall be easily adaptable to be integrated into an unmanned areal vehicle (UAV). The project result of SHARP-PS shall meet the following requirements:

Requirements for the UAV are not described in this document. These are dependent on the mission scenario for the UAV which, as of the writing of this document, has not yet been determined. This mission scenario will be determined by Stellar Space Industries as soon as possible, after which the requirements for the UAV will be constructed. Requirements for the other results of the SHARP-PS project are as follows:

\begin{itemize}
    \item Stellar Space Industries' FCU shall be included in the system.
    \item Solid Flow's hydrogen CGG shall be included in the system.
    \item A fuel cell shall be used to generate electricity from hydrogen.
    \item The system shall function without external power supply.
    \item The power density (power output over system weight) of the system must be the same or higher than that of state-off-the-art UAV power systems.
    \item Test results will be recorded and presented in a research report that is in accordance with `Report writing for readers with little time'~\cite{EllingAndeweg2012}
    \item The research report shall answer the following research question: `To what extent is a hydrogen fuel cell based power generation and feed system that integrates a Solid Flow's cool gas generator and Stellar Space Industries' flow control unit feasible for extending the mission time for UAV's'
\end{itemize}
\newpage

The aforementioned research question will be subdivided into the following sub research questions:

% \textbf{Sub Research Questions}\\
\begin{enumerate}
    \DTLforeach{researchquestions}{
    \id=id%
    }{
    \renewcommand{\labelenumi}{\id}
    \item{\RQ{\id}{long}}
          }
\end{enumerate}

These research questions have been arranged into a research matrix where they have been assigned a research method and is specified what instruments or tools are used and what the output is of each research question. This research matrix is shown in Appendix A.

\newpage


